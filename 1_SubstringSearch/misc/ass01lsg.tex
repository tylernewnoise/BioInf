\documentclass[a4paper,10pt]{article}
\usepackage[ngerman]{babel}		%dt. Übersetzung und Umlaute
\usepackage[utf8]{inputenc}		%Umlaute direkt eingeben
\usepackage{paralist}			%bessere Listen
\usepackage{listings}
\usepackage{listings}
\usepackage{fancyhdr}			%Headerstyles
\usepackage[margin=2.0cm,headheight=40pt,top=3cm]{geometry}
\pagestyle{fancy}

\lstset{
	basicstyle=\ttfamily,
	columns=fullflexible,
	frame=single,
	breaklines=true,
}
\lstdefinestyle{customc}{
	belowcaptionskip=1\baselineskip,
	title=\lstname,
	breaklines=true,
	keepspaces=true,
	flexiblecolumns=true,
	tabsize=2, % ein tab = 2 spaces
	numbers=left,
	frame=single,
	language=C,
	showstringspaces=false
}
\renewcommand{\headrulewidth}{0.4pt}
\renewcommand{\footrulewidth}{0.4pt}
\lhead{}
\rhead{Grundlagen der Bioinformatik - Assignment1}
\cfoot{}
\rfoot{\thepage}
\begin{document}
	\parindent0pt
	\paragraph{Analyse transcription factor GATA2}
	\begin{itemize}
		\item http://jaspar.genereg.net/matrix/MA0036.2/
		\item Formel: TODO
		\item Frequenzmatrix und info content TODO
		\item ic für eine Stelle berechnen TODO
		\item \begin{tabular}[t]{|c|p{10.5cm}|c|}
			\hline 
			Cancer & Title & PMID \\ 
			\hline 
			Prostate & GATA2 expression and biochemical recurrence following salvage radiation therapy for relapsing prostate cancer. & 28486040 \\ 
			\hline 
			Muscle/EBV related & Resolution of Multifocal Epstein-Barr Virus-Related Smooth Muscle Tumor in a Patient with GATA2 Deficiency Following Hematopoietic Stem Cell Transplantation. & 27924436 \\ 
			\hline 
			Leukemia & Clinical impact of GATA2 mutations in acute myeloid leukemia patients harboring CEBPA mutations: a study of the AML study group. & 27375010 \\ 
			\hline 
			Prostate & Whole Genomic Copy Number Alterations in Circulating Tumor Cells from Men with Abiraterone or Enzalutamide-Resistant Metastatic Castration-Resistant Prostate Cancer. & 27601596 \\ 
			\hline 
			Leukemia & Exome sequencing identifies highly recurrent somatic GATA2 and CEBPA mutations in acute erythroid leukemia. & 27389056 \\ 
			\hline 
			Leukemia & Acute lymphoblastic leukemia in a patient with MonoMAC syndrome/GATA2 haploinsufficiency. & 27232273 \\ 
			\hline 
			Muscle/EBV related & Association of GATA2 Deficiency With Severe Primary Epstein-Barr Virus (EBV) Infection and EBV-associated Cancers. & 27169477 \\ 
			\hline 
			Leukemia & Transcription factor gene GATA2: Association of leukemia and nonsynonymous to the synonymous substitution rate across five mammals. & 26850985 \\ 
			\hline 
		\end{tabular} 
	\end{itemize}
	
	\paragraph{Substring search}\ \\
	Wir haben für die Suche den Boyer-Moore-Horspool Algorithmus implementiert und somit eine durchschnittliche Laufzeit von $\mathcal{O}(\frac{n}{m})$ und im Worst-Case $\mathcal{O}(n \cdot m)$ .

	\paragraph{Properties of Boyer Moore Algorithm}
	\begin{enumerate}
		\item Template: \verb|aaaaaaaaaaaaa|\\
		Pattern: \verb|baaa|\\
		Bei diesem Beispiel kann die \textit{bad character rule} nur um einen Index weiterschieben und die \textit{good suffix rule} kommt überhaupt nicht zum Tragen. Es wird also jedes mal komplett durch die innere Schleife $n$ durchgelaufen, daher die Komplexität $\mathcal{O}(m*n)$. Verhindern ließe sich das mit der \textit{Rule of Galil}.
		\item Bei großen Alphabeten ist die Wahrscheinlichkeit höher, dass ein Zeichen gar nicht erst im Pattern ist, womit das Pattern durch die \textit{bad character rule} komplett geschoben werden kann. Besonders bei längeren Pattern ist dies von Vorteil. Es ist außerdem auch aufwendiger  die Shifts bei der \textit{good suffix rule} zu berechnen.
		
	\end{enumerate}


\end{document}
